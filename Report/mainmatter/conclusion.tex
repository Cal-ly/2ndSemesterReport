\chapter{Konklusion}
\label{chapter:conclusion}

\section{Opsummering}
\label{sec:conclusion_summary}

\subsection{Forretningsanalyse}
Gennem diverse modeller og analyse blev der grundlagt en solid forståelse af stakeholders egne ønsker, samt hvad projektet yderligere kunne bidrage af værdi. 
Ud fra dette skulle der udformes User Stories, som kunne bidrage til at projeket holdt kursen mod et godt produkt. 
Ydermere var der overvejet for hvilke dele og funktionaliteter, der først skulle flyttes out-of-scope, skulle der opstå modstand i løbet af processen.
Det blev vurderet at projektet var viable ud fra de stillede parametre og foranstaltninger. 

\subsection{SCRUM}
SCRUM processen blev brugt til at strukturere projektet og sikre, at der blev leveret funktionalitet til tiden.
Trello blev anvendt til at holde styr på processen og dokumentere arbejdet.
Der blev der brugt Workflows, der gav input til User Stories og Mapping. Hver User Story fik Tasks, Acceptance Criteria og et Story Point estimat.
Herfra blev der lavet en prioriteret Product Backlog, der blev brugt til at planlægge Sprints, som blev yderligere raffineret.
Gennem iterationer af Daily Standups, Sprint Reviews og Retrospectives - som var modificeret til omstændighederne - blev processen udviklet og optimeret.

\subsection{Software Design}
Der blev udarbejdet en række modeller, der beskrev systemets struktur og funktionalitet. Kernemodellerne har understøttet udviklingen af systemet og har været retningsgivende for implementeringen.
De har dog alle undergået en række ændringer i løbet af projektet, for at tilpasse sig systemets udvikling og de krav, der opstod undervejs.
Der blev lagt vægt på at designe et system, der var fleksibelt og kunne udvides med nye funktionaliteter. Bl.a. at en Employee har et Salary, der senere kan bruges i et lønsstyringsmodul.
Samtidig var brugervenlighed og det intuitive design indtænkt, så brugerne kunne finde rundt og bruge systemet uden problemer.
Hertil kom også overvejelser omkring sikkerhed og performance, der blev indtænkt i designet. Unit test blev anvendt i mindre omfang.
Den dokumentation, der blev udarbejdet, har været lavet med den hensigt, at en læser kan forstå kernedele af systemet og dets funktionalitet, uden at have været med i udviklingsprocessen.
Der er dog ikke fuld dokuemntation eller test, der beskriver alle aspekter af systemet, da det ville være for omfattende og ikke nødvendigt for projektet.
Prioriteringen er med henblik på, at have en enkelt udvikler, der har siddet med alle dele af projektet.
Der har dog været eksperimenteret med bl.a. Doxygen, for at kunne generere dokumentation.

\section{Diskussion}
\label{sec:conclusion_discussion}
Workflows fungerede ret godt, trods det opstod som en strøtanke. Prioriteringen og estimeringen af User Stories var dog mere tidskrævende end værdiskabende. 
Det kan skyldes at det var et kort og mindre omfangsrigt projekt, hvor det er nemmere at have overblik over, hvad der skal implementeres.
Sprints var en god måde at strukturere arbejdet på, da det gav en fornemmelse af, hvor meget der kunne implementeres i en uge, og dermed hvor meget der kunne implementeres i projektet som helhed.

SCRUM processen som den orginalt er beskrevet, er en fin metode med et godt værdisæt til at strukturere et projekt og sikre den rette kurs.
Den originale tanke tillader, at elementer kan skræddersys til den enkelte organisation eller projekt, og det er vigtigt at være opmærksom på, hvad der giver værdi og hvad der er spild af tid.
Dette sætter høje krav, ikke bare til at alle i teamet er bekendt med processen og er villige til at følge den, men også at teamet er samarbejdet og på bølgelængde.
Kravene til de individuelle teammedlemmer er også højere, da der kræves en vis grad af både selvstændighed og disciplin, men i den grad også kompetence og tværfaglighed for at kunne følge processen.
Underforstået, at hvis man fx som udvikler ikke har de fornødne kompetencer til at løse den øverste User Story i backloggen, så skævvrides processen og afliver tanken om "alle kan alt". 
Dette krav til kompetencer m.m. synes ikke beskrevet blandt de agile guruer. Det symptombehandles blot med mantraet om at man skal have det agile mindset, hvilket er ligeså formålstjenestligt som "Don't Panic".
Med tanke på den palette af kompetencer, der kræves, er det ikke underligt at SCRUM ofte bliver kritiseret for at være en metode, der er svær at implementere i praksis.
Ud fra egne erfaringer med at udvikle dette projekt alene, så har det stillet store krav til at kunne skifte mellem forskellige tværfaglige discipliner og teknologier. 
Alle kompetencer har været i spil og samtidig har der skulle erhverves nye eller generhverves uddaterede kompetencer, oveni at alle arbejdsopgaver kunne uddeles med et lommespejl.
Det er ikke en opgave, der er for alle, og det er ikke en opgave, der er for alle at løse alene.

Dokumentationen af systemet har været en balancegang mellem at have en forståelig og brugbar dokumentation og at have en dokumentation, der er så omfattende, at den tager for meget tid fra udviklingen.
Samtidig har der været en afvejning mellem at implementere en nye funktionalitet, afprøve den, muligvis skulle udarbejde tests og så at dokumentere den.
Det har været en udfordring at finde den rette balance, og det er ikke sikkert, at den er blevet fundet i dette projekt.  

\section{Konklusion}
\label{sec:conclusion_conclusion}
Det har været en lærerig proces, og det er klart, at der er plads til forbedring.

\section{Perspektivering}
\label{sec:conclusion_perspective}

\section{Afsluttende bemærkninger}
\label{sec:conclusion_remarks}

