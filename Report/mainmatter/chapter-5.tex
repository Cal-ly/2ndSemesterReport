\chapter{Software Design}
\label{chapter:software-design}

\section{Domain Model Diagram}
Det første skridt i designprocessen er at lave et Domain Model Diagram (DMD). Det er vigtigt at bemærke følgende:
\begin{itemize}
    \item DMD forholder sig til den virkelige verdens koncepter og relationer
    \item DMD er en statisk model, der ikke tager højde for ændringer i systemet over tid
    \item DMD anvender \emph{entities} til at repræsentere konceptuelle klasser
    \item DMD anvender \emph{relations} til at vise, hvordan \emph{entities} er relateret til og interagere med hinanden på et konceptuelt plan
    \item DMD kan bruges til at skabe en fælles forståelse af projektet og identificere de vigtigste klasser og \emph{relations}
    \item DMD kan bruges til at skabe en bedre forståelse af problemet for udviklerne og dermed en bedre løsning
    \item DMD kan bruges til at identificere de vigtigste klasser og deres \emph{relations}
    \item DMD er ikke en teknisk model og kan ikke direkte implementeres i software
\end{itemize} 
Kontrast skal primært ses i forhold til et Domain Class Diagram, hvor hver klasse har attributter, metoder og er forbundet andre klasser gennem specificerede relationer (fx agrregation) og mulitplicitet. 
Ligeledes ville man i et DCD se klasserne Basket, BasketItem og BasketSerice, muligvis med et IBasketService interface som implementeringen af konceptet "Kurv".
Det er vigtigt at definere ovenstående, da DMD er et godt værktøj men der er for mange holdninger til hvad det egentlig er.

\section{Database Design}
This is a reference to the book in the bibliography~\cite{connolly2023database}.

\section{Software Arkitektur}

\subsection{Klasse Diagram}

\subsection{Sekvens Diagrammer}

\section{Teknologier}

\section{Frameworks}

\section{Frameworks, Libraries and Packages}



\section{Design Patterns}

\section{Sikkerhed}

\section{Test}

\section{Dokumentation}