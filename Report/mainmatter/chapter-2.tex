\chapter{Udgangspunkt for projektet}
\label{chapter:udgangspunkt}

Den 11. april blev det originale team splittet op. Dette kapitel anvendes til at reflektere over forløbet op til opsplitningen, hvad der skete undervejs og efterfølgende, samt mine tilknyttede overvejelser.
Dette er altså ikke selve SCRUM processen, men udgangspunktet for en ny opstart. Kapitlet danner derfor grundlag for en ny projektplan, der vil blive udarbejdet i de næste kapiteler.

\section{Sprint Retrospektive}
For at holde SCRUM rammen, vil jeg bruge et Sprint retrospektive til at reflektere over opsplitningen.
Jeg tillader mig dog at tilføje punktet "hvad var planen" til "Hvad gik godt, hvad gik skidt og hvad gør vi til næste gang". Grundet det reflekterende persepktiv, skrives afsnittet i datid.  

\subsubsection{Hvad var planen}
Der blev med projektkontrakten sat en ramme for samarbejdet og teamet har tidligere arbejdet sammmen om semesterprojektet i første semester.
Der var en fælles, udtalt forståelse om et højt ambitionsniveau for projektet og SCRUM skulle være grundlaget for selve projektstyringen.
Arbejdsbyrden blev sat til at være indenfor normal undervisningstid på de gængse undervisningsdage (mandag-tirsdag og torsdag-fredag).

\subsubsection{Hvad gik godt}
Business analysen med bl.a. BMC og SWOT blev relativt hurtigt eksekveret.

\subsubsection{Hvad gik skidt}
Der var blandt alle medlemmer en eller anden form for utilfredshed med de andre i teamet, dog ikke noget der blev luftet åbent.      
Der var tillige friktioner mellem visse teammedlemmer fra sidste projekt, måske endda fra begyndelsen af studiet. Utilfredsheden samt friktionerne blev mere og mere tydelige og begyndte at fylde i samarbejdet. Eksempelvis blev det nævnt i en sidebemærkning under fire øjne, at det var frustrerende for et teammedlem, at det var sket mere end én gang, at den officielle frokostpause var sprunget over, fordi et andet teammedlem ikke medbragte eller spiste frokost og denne hellere ville gå tidligere hjem. 
Efter gruppedannelsen begyndte dele af teamet at forlange, at der kun måtte anvendes koncepter, syntax m.m. som var blevet undervist i eller havde været anvendt i opgaverne i semesteret Der var bl.a. en situation, hvor et forslag om at anvende denormalisering af databsem, ikke engang blive taget op i plenum, før en fra teamet kom med et person-orienteret udfald.
Friktionerne, muren i krogene, den passiv-aggressive tone m.m. satte sig yderligere i en form for mistillid, der skabte en åbenlys dårlig stemning for alle teammedlemmer. 
Dette miljø nedbrød enhver konstruktiv dialog og det førte til konklussionen, at et professionelt samarbejde ikke længere var en mulighed. 

Yderligere var der en meget forskellig opfattelse af hvordan SCRUM skulle anvendes. Der var teamets ønske, fra sidste projekt, om mere overordnet styring. Dette forsøgtes implementeret i dette projekt, da en SCRUM master fremlagde en overordnedet tidsplan for en dagen, hvor det var nødvendigt at hele teamet arbejdede sammen. Situationen udviklede sig desværre til et person-orienteret udfald, hvor et teammedlem fortalte SCRUM master, at SCRUM master skulle stoppe med at prøve på at være projektleder og bestemme over de andre teammedlemmer.

\subsubsection{Hvad gøres anderledes til næste gang}
Der kan spekuleres i, hvor mange af disse udfordringer ville kunne være blevet udglattet eller overhovedet vææe opstået, såfremt at den interpersonelle kemi i teamet havde været bedre. Jeg vil dog stadig nedfælde nogle punkter, der kan anvendes til fremtiden.

Et højt ambitionsniveau kan stå i kontrast til den begrænsede arbejdstid, som teamet fastsatte. Fremover bør der estimeres et mere konkret mål for et projekt og derefter afsættes tid eller vice versa, hvis ønsket om at holde studiet indenfor undervisningstiden vægtes højest.

Den overordnede ramme, i det her tilfælde SCRUM, bør også mere klart defineres. 
Selvom SCRUM var forsøgt defineret i dette projekt, bl.a. var det beskrevet, at SCRUM master skulle facilitere møderne, var der uenighed om hvad det at \emph{facilitere} indebar. 
Såfremt man vælger at have en SCRUM master, så bør der var en klar relation mellem ansvar, pligter og beføjelser. Såfremt SCRUM master skal være \emph{process owner}, bør der dertil være en klarhed om at SCRUM master bl.a. kan, bør og skal indkalde til møder og sætte dagsorden samt tidsplan for disse. Selv i velfungerende og sammenrystede teams, sammensat af højt motiverede og kompetente medlemmer, er der behov for en med det overordnede ansvar for processen. Både teori og erfaring peger mod at nye, uprøvede teams \cite{mit-newteams} kræver bl.a. klare mål og mere struktur, for at lykkedes og trives. Dette er yderligere understreget når de individuelle medlemmer har begrænset erfaring indenfor feltet, de skal arbejde i.

\section{Videreførsel}
Dette afsnit er skrevet kort efter opsplittelsen mhp. at fremstille tanker, planlægning m.m. der er foregået efterfølgende. 

\subsection{Opdeling af arv}
Det materiale der var udarbejdet i fællesskab, er fælles eje og store dele af det vil gå igen uden citat, bl.a. forretningsanalyse. Andre dele vil dog være markant omskrevet, således at de følger tråden i dette projekt. Rapporten, som den så ud 11. april, er vedlagt som bilag og kan blive refereret ved behov i denne rapport.

\subsection{SCRUM som enkeltmand}
Det kan anskues som kunstigt, at skulle videreføre SCRUM med alt hvad det indeholder, når man ikke har en Product Owner og er et team bestående af én. Visse SCRUM artefakter kan dog anvendes til projektstyring, bl.a. User Stories og Burndown Charts bevare sit oprindelige formål. Det vil undersøges hvorvidt en lektor kan indgå som Product Owner til nogle sessioner.

\subsection{Sprint planning}
Tidsrammen fra opsplitningen til deadline giver knap syv uger til at komme i mål. 
Det forventes, at der afsættes en uge til at sammenfatte og redigere eksisterende materiale, bl.a. omskrive og reestimere User Stories. Dette vil blive kaldt Sprint 0.
Den resterende tid deles op i fem sprints af én uges varighed og en aflsuttede halv uge. Hvert sprint sættes til 37 timer. Den afsluttende uge bruges til bl.a. at sammenfatte rapporten.