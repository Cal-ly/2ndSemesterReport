\chapter{Produktet}
\label{chapter:final-product}

\section{Programmet}
\label{sec:the-program}
Selve koden ligger offentligt tilgængelige på GitHub \cite{the-project}.

\section{Rapporten}
\label{sec:the-report}
\LaTeX til denne rapport ligger offentligt tilgængelige på GitHub \cite{the-report}.

\section{Dokumentation}
\label{sec:the-documentation}
Følger man linket \cite{the-auxiliary}, kan man finde videodemonstration af projektet, sammen med Doxygen dokumentationen for selve projektet. 

\section{Udviklerens kommentar}
\label{sec:the-developers-comment}
Projektet er blevet afsluttet til tiden, og der er blevet leveret et produkt, der opfylder de krav, der blev stillet i starten af projektet.
Der er dog nogle ting, der kunne være gjort anderledes, hvis der havde været mere tid. Designet er rudimentært men acceptabelt.
I forhold til funktionalitet, er der gået i breden fremfor dybden. Med tanke på at det er et skoleprojekt, er det dog acceptabelt, da det er vigtigere at kunne
implementere et solidt fundament af teknologier og teknikker, fremfor at enhver property og metode er suboptimeret og dokumenteret i en konsol app (subjektiv holdning).  
