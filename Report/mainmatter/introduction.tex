\chapter{Introduction}
\label{chapter:introduction}

\section{Case}
Blomsterbinderiet er en mindre butik, der ønsker en hjemmeside til at kunne servicere kunder online.
Deres produkt er primært blomsterdekorationer til forskellige begivenheder og af varierende størrelse.
Blomsterbinderiet har både walk-ins, bestillinger og et samarbejde med flere bedemænd i Roskilde og omegn. Til bedemænd er der særlige services, bl.a. levering til begravelser. Bestillinger fra private skal afhentes.
Selve butikken er fysisk placeret som en afdeling i Meny Roskilde og har ansat 3 blomsterdekoratører og et varierende (3-10) antal ufaglærte, hovedsageligt unge, assistenter.
Betalinger sker via MobilePay for private og via faktura for erhverv. Dette passer butikken godt, da de ofte gerne vil kunne kontakte kunden med spørgsmål, inden de acceptere ordren.
Ingen i butikken har IT erfaring og der er ikke budget til at have en fast IT supporter. Butiksbestyren, som vil være aftager af vores software, har talt om følgende:
\begin{itemize}
	\item Hjemmeside, hvor kunder kan afgive bestillinger
	\item Mulighed for at afvise bestillinger/slå bestillinger fra
	\item Mulighed for at kontakte kunden inden de acceptere en afgivet ordre
	\item Mulighed for et galleri til at inspirere kunderne
	\item Funktion hvor de kan se indkommende ordre evt notifiktaion
	\item Funktion hvor kunder kan kontakte butikken (evt. kontaktformular)
	\item Særlige funktionaliteter til bedemænd
	\item Kunden skal kunne vælge mellem forskellige størrelser af samme produkt
	\item Funktion, så de selv kan oprette, opdatere og arkivere produkter
	\item Mulighed for at få salgsstatestik, af endnu uspecificeret karatker.
	\label{list:butikkens-ønsker}
\end{itemize}

\section{Projektet}

\subsection{Problemformulering}
Hvordan kan man gennem Razor Pages, Entity Framework og med SCRUM som projektstyringsværktøj, udvikle en online webshop til BlomsterBinderiet, 
så den både imødekommer behovene hos almindelige kunder og lokale bedemænd gennem tilpassede brugeroplevelser og administrationsfunktioner.

\subsection{Formål}
Formålet med projektet er at udvikle en webshop til Blomsterbinderiet, der kan imødekomme behovene hos både almindelige kunder og lokale bedemænd. 
Dette skal ske gennem en brugervenlig og intuitiv brugergrænseflade, der giver kunderne mulighed for at afgive bestillinger, se produkter og kontakte butikken. 
Samtidig skal webshoppen give butikken mulighed for at administrere produkter, bestillinger og kunder, samt at se salgsstatistikker.

\subsection{Afgrænsning}
Projektet vil fokusere på udviklingen af en webshop til Blomsterbinderiet, der kan imødekomme behovene hos både almindelige kunder og lokale bedemænd. 
Der vil være mindre afvigelser fra hvad man ville gøre i virkeligheden, da projektet er et skoleprojekt og ikke et betalt kontraktprojekt. 
Dermed vil der være punkter, hvor fokus er på selve processen, dokumentation af denne og levere et produkt til skolastisk vurdering.

\subsection{Rapportens opbygning}
Rapporten vil begynde med projektets udgangspunkt og rammer i \Cref{chapter:udgangspunkt}, efterfulgt af en analyse af Blomsterbinderiets forretningsmodel i \Cref{chapter:forretningsanalyse}. 
Dette videreføres i \Cref{chapter:scrum} om SCRUM processen som projektstyringsværktøj og hvordan det anvendes i denne sammenhæng, bl.a. anvendelse af User Stories.
Herefter følger \Cref{chapter:software-design} om software design, der beskriver både den indledende analysen bag modeller og strukturer samt hvilke teknologier der tænkes er anvendt.
Dette vil føre over i \Cref{chapter:scrum-documentation} om selve udviklingen af projektet, hvor de enkelte Sprints dokumenteres. 
I \Cref{chapter:final-design} vil der være en gennemgang af projektets endelig implementering, beskrivelse og tankerne bag de vigtigste afvigelser fra den oprindelige plan.
Til sidst vil der i \Cref{chapter:conclusion} være en konklussion og perspektivering, der opsummerer projektet og dets resultater, samt giver et bud på hvordan et fremtidigt projekt kunne videreføres.
Bemærk, alle kapiteler opsummeres i \Cref{sec:conclusion_summary} med henblik på at give et samlet overblik over projektet, i stedet for at afrunde de individuelle kapitler med en opsummering. 
Tanken bag, er at give læseren en mulighed for at få et overblik over projektet, blot ved at læse \Cref{chapter:introduction} og derefter \Cref{chapter:conclusion}.

\subsection{Metode og teknologi}
\label{sec:metode-teknologi}
Razor Pages (RP) er valgt, da det er pensum. 
Det giver dog nogle fordele bl.a. fordi det er en relativt simpel side-fokuseret model, der letter organisering og udviklingen af en webapplikation, der kræver flere særskilte sider, bl.a. til produkter, fremvise analyser m.m. 
Gennem Entity Framework (EF) kan vi bruge Code First approach, der letter vores udvikling, ved bl.a. at simplificere implementering af mindre modifikationer og speeder processen op, 
da vi kan fokusere på C-sharp klasserne som udgangspunkt for vores database modellering og styring. Hertil understøtter EF Language Integrated Query (LINQ), der muliggøre at kode typesikrede queries direkte i C-sharp. 
Det gør koden mere læsbar, vedligeholdelsesvenlig og vi skal heller ikke manuelt administrere SQL-scripts (samt det reducerer risikoen for SQL-injektionsangreb). 
Alt i alt, så forenkler det administrationen af applikationens datalag. En SQL-database giver et struktureret miljø til lagring af data, hvilket er essentielt for produktkataloget, brugerinformation og ordrestyring i projektet. 
Den strukturerede tilgang sikrer dataintegritet og letter komplekse forespørgsler, der fx anvendes ved analyser. Vi er bundet til SQL gennem EF.
GitHub anvendes til versionskontrol for at forenkle udviklingen. Det giver et effektivt miljø til versionskontrol og projektstyring, hvilket gør det til et solidt valg til næsten alle software projekter. 
Yderligere giver det mulighed for at branche og teste nye funktionaliteter men stadig have et minimum viable product på master branch. 
Visual Studio 2022 og ASP.NET Core Integration: Integrationen mellem Razor Pages, Entity Framework, GitHub og Visual Studio 2022 giver et kraftfuldt udviklingsmiljø til test og implementering af projektet. 
Denne integrerede tilgang forenkler udviklingsprocessen, fra skrivning af kode til administration af databaser og implementering af applikationen.
Simply.com er valgt som webhotel på baggrund af projektets scope, samt at de tilbyder forholdsvist lave udgifter til drift.